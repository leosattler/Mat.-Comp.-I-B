%%%%%%%%%%%%%%%%%%%%%%%%%%%%%%%%%%%%%%%%%%%%%%%%%%%%%%%%%%%%%%%%%%%%%%%%%%%%%%%%
% ABSTRACT


\begin{abstract}

%% neste arquivo abstract.tex
%% o texto do resumo e as palavras-chave têm que ser em Inglês para os documentos escritos em Português
%% o texto do resumo e as palavras-chave têm que ser em Português para os documentos escritos em Inglês
%% os nomes dos comandos \begin{abstract}, \end{abstract}, \keywords e \palavrachave não devem ser alterados

\selectlanguage{english}	%% para os documentos escritos em Português
%\selectlanguage{portuguese}	%% para os documentos escritos em Inglês

\hypertarget{estilo:abstract}{} %% uso para este Guia

Europa, one of Jupiter's satellites, is one of the best bets for extraterrestrial life in the Solar System. The \textit{Galileo} spacecraft's magnetometer revealed the presence of a conductive layer under Europa's icy crust, best explained as a salty water ocean. Also, \textit{Voyager} and \textit{Galileo} missions provided images and data that revealed an active and young surface, with few impact craters and several geological formations such as domes, pits, and evidence for place tectonics. Among these features, one is more pervasive and intriguing: the double ridges. They run from tens to hundreds of kilometers, with widths varying from few hundreds to 4 km. Ridge height is around tens of meters, with some examples reaching $\sim$400 m high. They overlap one another and appear in pairs as well as in more complex systems, with several lineaments parallel to each other. Several formation ideas have been explored, some linking existing cracks in the outermost shell to the subsurface water via mechanisms of heat transport and tectonism. Understanding double ridge formation will shine a light on the thermal evolution of Europa, and constrain physical parameters necessary for future missions for life exploration. This work is a review of the most promising and enticing proposals of double ridge formation, and discuss their weaknesses and strengths with regards to our current knowledge of the icy moon.

\keywords{%
	\palavrachave{Europa}%
	\palavrachave{Tectonics}%
	\palavrachave{Surface}%
	\palavrachave{Satellites of Jupiter}%
	\palavrachave{Double ridges}%
}

%\selectlanguage{portuguese}	%% para os documentos escritos em Português
\selectlanguage{english}	%% para os documentos escritos em Inglês

\end{abstract}